
%% README.md
%% Copyright 2017 Evandro Coan
%
% This work may be distributed and/or modified under the
% conditions of the LaTeX Project Public License, either version 1.3
% of this license or (at your option) any later version.
% The latest version of this license is in
%   http://www.latex-project.org/lppl.txt
% and version 1.3 or later is part of all distributions of LaTeX
% version 2005/12/01 or later.
%
% This work has the LPPL maintenance status `maintained'.
%
% The Current Maintainer of this work is M. Y. Name.
%
% This work consists of the files `README.md`, `basic.tex`,
% `commands.tex` and `programming.tex`.
\makeatletter



% Incompatible color definition when using tikz with color package
% https://tex.stackexchange.com/questions/150369/incompatible-color-definition-when-using-tikz-with-color-package
\usepackage{xcolor}

% RGB colors in absolute values from 0 to 255 by using `RGB` tag
\definecolor{darkblue}{RGB}{26,13,178}

% RGB colors in percentage notation by using `rgb` tag
\definecolor{darkgreen}{rgb}{0,0.6,0}
\definecolor{gray}{rgb}{0.5,0.5,0.5}
\definecolor{mauve}{rgb}{0.58,0,0.82}

\@ifpackageloaded{changepage}{}{\usepackage[backref,colorlinks,linkcolor=darkblue,citecolor=darkgreen]{hyperref}}


% pdfTeX warning (ext4): destination with the same identifier (name{figure.1}) has been already used, duplicate ignored
% https://tex.stackexchange.com/questions/124520/pdftex-warning-ext4-destination-with-the-same-identifier-namefigure-1-has
\usepackage{aliascnt,ccaption}

% Fix `destination with the same identifier (name{figure.1}) has been already used, duplicate ignored`,
% when using \caption{title} inside a figure.
%
% New alias counter for figure float
% http://ctan.org/pkg/aliascnt
% http://ctan.org/pkg/ccaption
\newaliascnt{figurealt}{figure}
\aliascntresetthe{figurealt}
\renewcommand{\contcaption}
{
    \expandafter\addtocounter\expandafter{\@captype alt}{\m@ne} % Step alias cntr back
    \expandafter\refstepcounter\expandafter{\@captype alt} % Make reference
    \@contcaption\@captype
}

% \lettrine{O}{nce} upon a time...
% \lettrine[findent=2pt]{\fbox{\textbf{T}}}{ }his thesis deals with...
%
% https://tex.stackexchange.com/questions/164298/starting-a-paragraph-with-a-big-letter
\usepackage{lettrine}

% Required for including pictures, resizebox
\usepackage{graphicx}

% Allows putting an [H] in \begin{figure} to specify the exact location of the figure
% https://tex.stackexchange.com/questions/8625/force-figure-placement-in-text
\usepackage{float}

% Allows in-line images such as the example fish picture
\usepackage{wrapfig}

% How to automatically force latex to not justify the text when it is not wise?
% https://tex.stackexchange.com/questions/365801/how-to-automatically-force-latex-to-not-justify-the-text-when-it-is-not-wise
\usepackage{array,ragged2e}

% Use its macro adjustwidth* to extend tables out of outer text border.
% https://tex.stackexchange.com/questions/366155/how-to-write-a-table-a-little-larger-than-the-paragraphs-with-centered-columns
%
% The `memoir` class emulates this package, so not try to load it when using `memoir`: ifpackageloaded question
% https://tex.stackexchange.com/questions/70212/ifpackageloaded-question
\@ifpackageloaded{changepage}{}{\usepackage[strict]{changepage}}

% With the package option shortlabels you can use an enumerate-like syntax, where A, a, I, i and 1
% stand for \Alph*, \alph*, \Roman*, \roman* and \arabic*. This is intended mainly as a sort of
% compatibility mode with the enumerate package, and therefore the following special rule applies:
% if the very first option (at any level) is not recognized as a valid key, then it will be
% considered a label with the enumerate-like syntax.
%
% No spacing between enumerated items with \usepackage{enumerate}
% https://tex.stackexchange.com/questions/119919/no-spacing-between-enumerated-items-with-usepackageenumerate
\@ifpackageloaded{changepage}{}{\usepackage[shortlabels]{enumitem}}
\@ifpackageloaded{changepage}{}{\usepackage[hyphens]{url}}

\usepackage{tabularx}
\usepackage{multirow}



% The package supports the Text Companion fonts, which provide many text symbols (such as
% baht, bullet, copyright, musicalnote, onequarter, section, and yen), in the TS1 encoding.
%
% `LaTeX Error: Option clash for package textcomp` with the package `mathcomp`, it need to be loaded before it.
% https://tex.stackexchange.com/questions/343546/how-do-you-resolve-the-error-in-latex-option-clash-for-package-inputenc-usepa
\usepackage[full]{textcomp}
\usepackage{mathcomp}

% Manipulação de Strings
\usepackage{xstring}

% Número da última página
\usepackage{lastpage}

% Tamanho das fontes
\usepackage{anyfontsize}

% Usa a fonte Latin Modern
\usepackage{lmodern}

% Selecao de codigos de fonte
% https://tex.stackexchange.com/questions/664/why-should-i-use-usepackaget1fontenc
%
% Will allow all displayable utf8 characters to be available as input
% https://tex.stackexchange.com/questions/13067/utf8x-vs-utf8-inputenc
\usepackage[T1]{fontenc}

% Codificacao do documento (conversão automática dos acentos)
\usepackage[utf8]{inputenc}

\usepackage{graphicx}  % Inclusão de gráficos

\usepackage{pdfpages}
\usepackage{fancyvrb}
\usepackage{rotating}
\usepackage{amsmath}
\usepackage{amssymb}
\usepackage{mathrsfs}
\usepackage{pdflscape}
\usepackage{epstopdf}
\usepackage{multirow}
\usepackage{listings}

% Para incluir links
\usepackage{url}

% Pacote necessário para a lista de siglas.
\usepackage{nomencl}
\usepackage{booktabs}

% A comprehensive (SI) units package
\usepackage{siunitx}
\sisetup{detect-all}
\sisetup{scientific-notation = fixed, fixed-exponent = 0, round-mode = places,round-precision = 2,output-decimal-marker = {,} }
\DeclareSIUnit \VA {VA} % apparent power

% Memoir class conflict with datetime
% https://tex.stackexchange.com/questions/162353/memoir-class-conflict-with-datetime
% https://tex.stackexchange.com/questions/49071/difference-between-let-foo-relax-and-def-foo-for-disabling
\let\ordinal\relax
\usepackage{datetime}

% gives you the possibility to rotate any object of an arbitrary angle.
\usepackage{rotating}

% Rotação de páginas no documento pdf.
\usepackage{pdflscape}

% Customize the look of the frame
\usepackage{mdframed}

% Pacotes adicionais, usados apenas no âmbito do Modelo Canônico do abnteX2
\usepackage{tablefootnote}
\usepackage{longtable}
\usepackage{ragged2e}
\usepackage{tocloft}

% LaTeX not hyphenating properly, text running off page
% https://tex.stackexchange.com/questions/28136/latex-not-hyphenating-properly-text-running-off-page
\usepackage{hyphenat}

% How to auto adjust my last table column width, and why is there Underfull \vbox badness on this table?
% https://tex.stackexchange.com/questions/387238/how-to-auto-adjust-my-last-table-column-width-and-why-is-there-underfull-vbox/387251
\usepackage{ltablex}
\keepXColumns

% How to use \scalebox around my environment?
% https://tex.stackexchange.com/questions/387515/how-to-use-scalebox-around-my-environment
\usepackage{verbatimbox}

% LaTeX/Indexing
% https://www.sharelatex.com/learn/Indices
% https://en.wikibooks.org/wiki/LaTeX/Indexing
\usepackage{makeidx}
\makeindex

% Is it possible to keep my translation together with original text?
% https://tex.stackexchange.com/questions/5076/is-it-possible-to-keep-my-translation-together-with-original-text
\usepackage{comment}



\makeatother


