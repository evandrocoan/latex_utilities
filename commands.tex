
%% README.md
%% Copyright 2017 Evandro Coan
%
% This work may be distributed and/or modified under the
% conditions of the LaTeX Project Public License, either version 1.3
% of this license or (at your option) any later version.
% The latest version of this license is in
%   http://www.latex-project.org/lppl.txt
% and version 1.3 or later is part of all distributions of LaTeX
% version 2005/12/01 or later.
%
% This work has the LPPL maintenance status `maintained'.
%
% The Current Maintainer of this work is M. Y. Name.
%
% This work consists of the files:
% 1. `README.md`,
% 2. `basic.tex`,
% 3. `commands.tex`,
% 4. `commands_list.tex`
% 5. `programming.tex`
% 6. `badboxes.tex`



%
% Settings
%

% RGB colors in absolute values from 0 to 255 by using `RGB` tag
\definecolor{darkblue}{RGB}{26,13,178}

% RGB colors in percentage notation by using `rgb` tag
\definecolor{darkgreen}{rgb}{0,0.6,0}
\definecolor{gray}{rgb}{0.5,0.5,0.5}
\definecolor{mauve}{rgb}{0.58,0,0.82}

% Basic settings for hyperref package
\hypersetup{colorlinks,linkcolor=darkblue,citecolor=darkgreen}



% How could the `\everypar` justification statement be used?
% https://tex.stackexchange.com/questions/365818/how-could-the-everypar-justification-statement-be-used
\newbox\linebox \newbox\snapbox
\def\eatlines
{%
    \setbox\linebox\lastbox % check the last line
    \ifvoid\linebox
    \else % if it’s not empty
        \unskip\unpenalty % take whatever is
        {\eatlines} % above it;
        \setbox\snapbox\hbox{\unhcopy\linebox}
        \ifdim\wd\snapbox<.98\wd\linebox
            \box\snapbox % take the one or the other,
        \else \box\linebox \fi
    \fi
}



% How could the `\everypar` justification statement be used?
% https://tex.stackexchange.com/questions/365818/how-could-the-everypar-justification-statement-be-used
\everypar=
{%
    \setbox0=\lastbox \par%
    \vbox\bgroup \everypar={}\def\par{\endgraf\eatlines\egroup}%
}

% Creates a new environment which can be used as:
%
% \begin{foo}
%   Text...
%
%   Text ...
% \end{foo}
%
% https://tex.stackexchange.com/questions/62333/push-long-words-in-a-new-line
\newenvironment{foo}
{%
    \par%
    \hyphenpenalty=10000%
    \exhyphenpenalty=10000%
}
{\par}



% Some text \brkurl{http://www.example.com/this/directory/here}
%
% How to break long URLs using common hyphenation but adding a line feed indicator?
% https://tex.stackexchange.com/questions/69824/how-to-break-long-urls-using-common-hyphenation-but
\def\addurlspace#1
{%
    \ifx\relax#1%
    \else%
    \ifx/#1\space\fi%
    \ifx.#1\space\fi%
    #1%
    \ifx/#1\space\fi%
    \ifx.#1\space\fi%
    \expandafter\addurlspace%
    \fi%
}

\makeatletter
\@namedef{OT1-zwidthchar}{255}
\@namedef{T1-zwidthchar}{"17}

\def\brkurl#1
{%
    \edef\savedhchar{\the\hyphenchar\font}%
    \global\setbox1\hbox{}%
    \setbox0=\vbox
    {%
        \hsize=2pt\rightskip=0pt plus 1fill%
        \hfuzz\maxdimen%
        \tracinglostchars0%
        \overfullrule0pt%
        \hyphenchar\font=\csname \f@encoding-zwidthchar\endcsname%
        \noindent \hskip0pt \addurlspace #1\relax%
        \par%
        \loop%
        \setbox4 \lastbox%
        \ifvoid4 \else%
        \global\setbox1\hbox%
        {%
            \unhbox4\unskip\unskip\discretionary{\hbox{\rlap{$\leftarrow$}}}{}{}\unhbox1%
        }%
        \unskip%
        \unskip%
        \unpenalty%
        \unskip%
        \repeat%
    }%
    \unhbox1%
    \hyphenchar\font\savedhchar%
    \relax%
}
\makeatother



% Change background color for text block
% https://tex.stackexchange.com/questions/238294/change-background-color-for-text-block
\usepackage{framed}
\usepackage[most]{tcolorbox}
\definecolor{shadecolor}{RGB}{219, 229, 241}
\newtcolorbox{bluebox}
{%
    colback=shadecolor,
    grow to right by=-2mm,
    grow to left by=-2mm,
    boxrule=0pt,
    boxsep=0pt,
    breakable,
}



% Make first row of table all bold
%
% Usage:
% 1. Add `B` on the borders and `^` before each column definition.
% 2. `\rowstyle{\bfseries}` before the row you want to bold.
%
% Example:
% \begin{tabularx}{\linewidth}
% {|
%     *1{                 >{\RaggedRight\arraybackslash\hsize=1.1\hsize }BX       |} % Riscos
%     *3{@{\hspace{3.0pt}}>{\Centering\arraybackslash                   }^p{0.9cm}|} % Probabilidade, Impacto, Prioridade
%     *2{                 >{\RaggedRight\arraybackslash\hsize=0.95\hsize}^X       |} % Resposta, Prevenção
% }
%
% \hline
%
% \rowstyle{\bfseries}
% Riscos  & 1 & 2 & 3 & Estratégia de resposta & Ações de prevenção \\ \hline
%
%
% https://tex.stackexchange.com/questions/4811/make-first-row-of-table-all-bold
\usepackage{array}
\newcolumntype{B}{>{\global\let\currentrowstyle\relax}}
\newcolumntype{^}{>{\currentrowstyle}}
\newcommand{\rowstyle}[1]
{%
    \gdef\currentrowstyle{#1}#1\ignorespaces
}


% https://tex.stackexchange.com/questions/485834/why-abntex2-class-is-inserting-a-new-line-after-the-chapter-title
% https://tex.stackexchange.com/questions/367859/how-to-automatically-put-a-go-to-summary-go-back-on-each-section
% https://tex.stackexchange.com/questions/388045/how-can-the-go-to-summary-be-fixed-so-the-sectionsomesome-more
% https://tex.stackexchange.com/questions/399635/what-is-the-equivalent-to-sectionformat-on-memoir-class-for-chapterformat
\definecolor{ultramarine}{RGB}{0,32,96}
\RequirePackage{xpatch}
\RequirePackage{amssymb}
\RequirePackage{hyperref}
\newcommand{\goToSummaryText}
{%
    \small\mdseries
    \hyperlink{summary}{\textcolor{ultramarine}{$\leftleftarrows$}}
    {$|$}
    \Acrobatmenu{GoBack}{\textcolor{ultramarine}{$\leftarrow$}}
}
\makeatletter
    \newif\ifismemoirloaded\ismemoirloadedfalse
    \newif\ifisabntexloaded\isabntexloadedfalse
    \@ifclassloaded{memoir}{%
        \ismemoirloadedtrue%
    }{}
    \@ifclassloaded{abntex2}{%
        \isabntexloadedtrue%
    }{}
    \newcommand{\addGoToSummary}
    {%
        \@ifundefined{printpartnum}{\message{printpartnum patch for addGoToSummary could NOT
            be applied because there is no printpartnum command available!^^J}}{%
            \let\oldAddGoToprintpartnum\printpartnum
            \xapptocmd{\printpartnum}{~\protect\goToSummaryText}{}{}
        }
        \@ifundefined{Sectionformat}{\message{Sectionformat patch for addGoToSummary could NOT
            be applied because there is no Sectionformat command available!^^J}}{%
            \let\oldAddGoToSectionformat\Sectionformat
            \xapptocmd{\Sectionformat}{~\protect\goToSummaryText}{}{}
        }
        \ifismemoirloaded
            \ifisabntexloaded
                \let\oldAddGoToABNTEXchapterupperifneeded\ABNTEXchapterupperifneeded
                \xapptocmd{\ABNTEXchapterupperifneeded}{~\protect\goToSummaryText}{}{}
            \else
                \let\oldAddGoToprintchaptertitle\printchaptertitle
                \xapptocmd{\printchaptertitle}{~\protect\goToSummaryText}{}{}
            \fi
        \else
            \@ifundefined{Chapterformat}{\message{Chapterformat patch for addGoToSummary could NOT
                be applied because there is no Chapterformat command available!^^J}}{%
                \let\oldAddGoToChapterformat\Chapterformat
                \xapptocmd{\Chapterformat}{~\protect\goToSummaryText}{}{}
            }
        \fi
    }
    \newcommand{\removeGoToSummary}
    {%
        \@ifundefined{printpartnum}{\message{printpartnum patch for removeGoToSummary could NOT
            be applied because there is no printpartnum command available!^^J}}{%
            \let\printpartnum\oldAddGoToprintpartnum
        }
        \@ifundefined{Sectionformat}{\message{Sectionformat patch for removeGoToSummary could NOT
            be applied because there is no Sectionformat command available!^^J}}{%
            \let\Sectionformat\oldAddGoToSectionformat
        }
        \ifismemoirloaded
            \ifisabntexloaded
                \let\ABNTEXchapterupperifneeded\oldAddGoToABNTEXchapterupperifneeded
            \else
                \let\printchaptertitle\oldAddGoToprintchaptertitle
            \fi
        \else
            \@ifundefined{Chapterformat}{\message{Chapterformat patch for removeGoToSummary could NOT
                be applied because there is no Chapterformat command available!^^J}}{%
                \let\Chapterformat\oldAddGoToChapterformat%
            }
        \fi
    }
\makeatother
\let\oldAddGoTotableofcontents\tableofcontents
% Insert internal document link
\renewcommand{\tableofcontents}{%
    \hypertarget{summary}%
    \oldAddGoTotableofcontents%
}



%
% New commands
%

% Allow to push long words on new lines when they do not fit entirely on the current line.
% https://tex.stackexchange.com/questions/62333/push-long-words-in-a-new-line
\newcommand\lword[1]{\leavevmode\nobreak\hskip0pt plus\linewidth\penalty50\hskip0pt plus-\linewidth\nobreak{#1}}
\newcommand\lurl[1]{\leavevmode\nobreak\hskip0pt plus\linewidth\penalty50\hskip0pt plus-\linewidth\nobreak{\url{#1}}}


% For the new command \latex
\usepackage{xspace}

% Write the word LaTeX nicely.
\newcommand{\latex}{\LaTeX\xspace}

% Create a bold title all in upper case.
\newcommand{\Title}[1]{\textbf{\MakeUppercase{#1}}}

% \nameref — How to display section name AND its number
% https://tex.stackexchange.com/questions/121865/nameref-how-to-display-section-name-and-its-number
%
% Usage \fullref{fig:envinronmentHead} or \fullref{sec:some_sec}
\newcommand*{\fullref}[1]{\hyperref[{#1}]{\autoref*{#1} \nameref*{#1}}} % One single link



% Setting Entries of List of Listings in LaTeX. Package Listings
% http://tex.stackexchange.com/questions/228936/setting-entries-of-list-of-listings-in-latex-package-listings
\makeatletter
\@ifclassloaded{memoir}{%
    \newlength\mylen

    % https://tex.stackexchange.com/questions/485830/why-latex-does-not-tell-me-which-command-is-undefined
    \@ifpackageloaded{babel}{\@ifpackagewith{babel}{brazil}{\addto\captionsbrazil{\renewcommand{\lstlistingname}{Código}}}{}}{}
    \@ifpackageloaded{babel}{\@ifpackagewith{babel}{brazil}{\addto\captionsbrazil{\renewcommand{\lstlistlistingname}{Lista de códigos}}}{}}{}

    \@ifpackageloaded{babel}{\@ifpackagewith{babel}{english}{\addto\captionsenglish{\renewcommand{\lstlistingname}{Code}}}{}}{}
    \@ifpackageloaded{babel}{\@ifpackagewith{babel}{english}{\addto\captionsenglish{\renewcommand{\lstlistlistingname}{List of Codes}}}{}}{}

    \begingroup
    \makeatletter
        \let\newcounter\@gobble\let\setcounter\@gobbletwo
        \globaldefs\@ne \let\c@loldepth\@ne
        \newlistof{listings}{lol}{\lstlistlistingname}
        \newlistof{lstlistoflistings}{lol}{\lstlistlistingname}
        \newlistentry{lstlisting}{lol}{0}
    \makeatother
    \endgroup

    % Why the empty space size is increasing each call to my calculate listing header command?
    % https://tex.stackexchange.com/questions/388411/why-the-empty-space-size-is-increasing-each-call
    \newlength\cftlstlistingoldnumwidth
    \setlength\cftlstlistingoldnumwidth{\cftlstlistingnumwidth}

    % Calculate the size of the header
    %
    % What is the use of percent signs (%) at the end of lines?
    % https://tex.stackexchange.com/questions/7453/what-is-the-use-of-percent-signs-at-the-end-of-lines
    \newcommand{\calculatelisteningsheader}
    {%
        \renewcommand\cftlstlistingpresnum{\lstlistingname~}%
        \settowidth\mylen{\cftlstlistingpresnum\cftlstlistingaftersnum}%
        \setlength\cftlstlistingnumwidth{\dimexpr\cftlstlistingoldnumwidth+\mylen}%
        \renewcommand\cftlstlistingaftersnum{\hfill\textendash\hfill}%
    }

    % Ensure it is called at least one time
    \calculatelisteningsheader

    % https://tex.stackexchange.com/questions/14135/how-to-automatically-add-text-immediately-after-begindocument
    \AtBeginDocument{\calculatelisteningsheader}
}{}
\makeatother



% How to create my own list of things
% https://tex.stackexchange.com/questions/61086/how-to-create-my-own-list-of-things
\newcommand{\mytextpreliminarylistname}{Short Table of Contents}
\newlistof{textpreliminarycontents}{tpc}{\mytextpreliminarylistname}

% Resetting counter
% https://tex.stackexchange.com/questions/66604/resetting-counter
%
% Custom list throw LaTeX Error: Command \mycustomfiction already defined?
% https://tex.stackexchange.com/questions/388489/custom-list-throw-latex-error-command-mycustomfiction-already-defined
\newlistentry{textpreliminarycounter}{tpc}{0}

% Continuing Page Numbering (Roman to Arabic)
% https://tex.stackexchange.com/questions/56131/continuing-page-numbering-roman-to-arabic
\renewcommand{\thetextpreliminarycounter}{\arabic{textpreliminarycounter}}

% Reset section numbering between unnumbered chapters
% https://tex.stackexchange.com/questions/71162/reset-section-numbering-between-unnumbered-chapters
\newcommand{\addtotextpreliminarycontent}[1]
{%
    \refstepcounter{textpreliminarycounter}%
    \addcontentsline{tpc}{textpreliminarycounter}{\protect\numberline{\thetextpreliminarycounter}#1}\par%
}



% Unable to link to inserted pages with pdfpages
% Solution from http://tex.stackexchange.com/questions/25105/unable-to-link-to-inserted-pages-with-pdfpages
\newcounter{includepdfpage}
\newcounter{currentpagecounter}

\newcommand{\addlabelstoallincludedpages}[1]
{%
    \refstepcounter{includepdfpage}%
    \stepcounter{currentpagecounter}%
    \label{#1.\thecurrentpagecounter}%
}
\newcommand{\modifiedincludepdf}[4]
{%
    \setcounter{currentpagecounter}{0}%
    \includepdf[pages=#1,pagecommand=\addlabelstoallincludedpages{#2},scale=#4]{#3}%
}



% \MakeUppercase in \section and \chapter with hyperref cause trouble
% https://tex.stackexchange.com/questions/199374/makeuppercase-in-section-and-chapter-with-hyperref-cause-trouble
\newcommand{\HyperrefUppercase}[1]{\texorpdfstring{\MakeTextUppercase{#1}}{#1}}



% Example about hyphenation with ttfamily font
% https://tex.stackexchange.com/questions/386665/example-about-hyphenation-with-ttfamily-font
%
% Why the environment ttfamily is hyphenated, but macro ttfamily is not hyphenating?
% https://tex.stackexchange.com/questions/387678/why-the-environment-ttfamily-is-hyphenated-but-macro-ttfamily-is-not-hyphenatin
\LetLtxMacro\origttfamily\ttfamily
\DeclareRobustCommand*{\ttfamily}
{%
    \origttfamily
    \hyphenchar\font=`\-\relax
    \fontdimen3\font=.25em\relax
    \fontdimen4\font=.13em\relax
    \fontdimen7\font=.167em\relax
}

\makeatletter
\DeclareRobustCommand\vttfamily
{%
    \not@math@alphabet\vttfamily\relax
    \fontfamily{cmvtt}% cmvtt (Computer Modern) or lmvtt (Latin Modern)
    \selectfont
}
\DeclareTextFontCommand{\textvtt}{\vttfamily}
\makeatother



% Logical String Length
% https://tex.stackexchange.com/questions/87638/logical-string-length
\newcommand{\includeonlyfilelist}[0]{}
\makeatletter
\newcommand{\addtoincludeonly}[1]
{%
    \StrLen{\includeonlyfilelist}[\includeonlyfilelistlen]

    % How to concatenate strings into a single command?
    % https://tex.stackexchange.com/questions/74707/how-to-concatenate-strings-into-a
    \ifnum\includeonlyfilelistlen>0
        \g@addto@macro\includeonlyfilelist{,#1}
    \else
        \g@addto@macro\includeonlyfilelist{#1}
    \fi
}
\newcommand{\doincludeonly}[0]
{%
    \StrLen{\includeonlyfilelist}[\includeonlyfilelistlen]
    \ifnum\includeonlyfilelistlen>0
        \includeonly{\includeonlyfilelist}
    \else
    \fi
}
\makeatother


